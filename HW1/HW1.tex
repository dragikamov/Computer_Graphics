\documentclass[a4paper]{article}
\usepackage[pdftex]{hyperref}
\usepackage[latin1]{inputenc}
\usepackage[english]{babel}
\usepackage{a4wide}
\usepackage{amsmath}
\usepackage{amssymb}
\usepackage{algorithmic}
\usepackage{algorithm}
\usepackage{ifthen}
\usepackage{listings}
% move the asterisk at the right position
\lstset{basicstyle=\ttfamily,tabsize=4,literate={*}{${}^*{}$}1}
%\lstset{language=C,basicstyle=\ttfamily}
\usepackage{moreverb}
\usepackage{palatino}
\usepackage{multicol}
\usepackage{tabularx}
\usepackage{comment}
\usepackage{verbatim}
\usepackage{color}

% Because of an error on line 41 I added this
\usepackage{graphicx}

% Used for drawing DFAs and NFAs
\usepackage{tikz}
\usetikzlibrary{automata, positioning}

% Defined checkmark sign
\def\checkmark{\tikz\fill[scale=0.4](0,.35) -- (.25,0) -- (1,.7) -- (.25,.15) -- cycle;}

% Table coloring
\usepackage{color, colortbl}
\usepackage[first=0,last=9]{lcg}

% Tab
\newcommand\tab[1][1.15cm]{\hspace*{#1}}

% Degree sign
\usepackage{gensymb}

% Matrices
\usepackage{amsmath}

%% pdflatex?
\newif\ifpdf
\ifx\pdfoutput\undefined
\pdffalse % we are not running PDFLaTeX
\else
\pdfoutput=1 % we are running PDFLaTeX
\pdftrue
\fi
%\ifpdf
%\usepackage[pdftex]{graphicx}
%\else
%\usepackage{graphicx}
%\fi
\ifpdf
\DeclareGraphicsExtensions{.pdf, .jpg}
\else
\DeclareGraphicsExtensions{.eps, .jpg}
\fi

\parindent=0cm
\parskip=0cm

\setlength{\columnseprule}{0.4pt}
\addtolength{\columnsep}{2pt}

\addtolength{\textheight}{5.5cm}
\addtolength{\topmargin}{-26mm}
\pagestyle{empty}

%%
%% Sheet setup
%% 
\newcommand{\coursename}{Computer Graphics}
\newcommand{\courseno}{CO19-320322}
 
\newcommand{\sheettitle}{Homework}
\newcommand{\mytitle}{}
\newcommand{\mytoday}{March 22nd, 2019}

% Current Assignment number
\newcounter{assignmentno}
\setcounter{assignmentno}{1}

% Current Problem number, should always start at 1
\newcounter{problemno}
\setcounter{problemno}{1}

%%
%% problem and bonus environment
%%
\newcounter{probcalc}
\newcommand{\exercise}[2]{
  \pagebreak[2]
  \setcounter{probcalc}{#2}
  ~\\
  {\large \textbf{Exercise \arabic{assignmentno}.\arabic{problemno}} \hspace{0.2cm}\textit{#1}} \refstepcounter{problemno}\vspace{2pt}\\}

\newcommand{\bonus}[2]{
  \pagebreak[2]
  \setcounter{probcalc}{#2}
  ~\\
  {\large \textbf{Bonus Problem \textcolor{blue}{\arabic{assignmentno}}.\textcolor{blue}{\arabic{problemno}}} \hspace{0.2cm}\textit{#1}} \refstepcounter{problemno}\vspace{2pt}\\}

%% some counters  
\newcommand{\assignment}{\arabic{assignmentno}}

%% solution  
\newcommand{\solution}{\pagebreak[2]{\bf Solution:}\\}

%% Hyperref Setup
\hypersetup{pdftitle={Homework \assignment},
  pdfsubject={\coursename},
  pdfauthor={},
  pdfcreator={},
  pdfkeywords={Computer Graphics},
  %  pdfpagemode={FullScreen},
  %colorlinks=true,
  %bookmarks=true,
  %hyperindex=true,
  bookmarksopen=false,
  bookmarksnumbered=true,
  breaklinks=true,
  %urlcolor=darkblue
  urlbordercolor={0 0 0.7}
}

\begin{document}
\coursename \hfill Course: \courseno\\
Jacobs University Bremen \hfill \mytoday\\
Dragi Kamov\hfill
\vspace*{0.3cm}\\
\begin{center}
{\Large \sheettitle{} \assignment\\}
\end{center}

\exercise{}{0}
\solution
\begin{enumerate}
    \item Every Cartesian coordinate system is orthogonal. Every orthogonal coordinate system is parallel. \textbf{True}
    \item Non-uniform scaling transformation is linear. \textbf{True}
    \item Order of transformations does not matter if one uses homogeneous coordinates. \textbf{False}
    \item Perspective projection in OpenGL yields vertices coordinates in (x, y) such that x $ \in $ $ \langle $0, screen\_width -1$ \rangle $, x $\in$ $ \langle $0, screen\_height - 1$ \rangle $. \textbf{True}
    \item OpenGL functions starting with $ glu $ prefix are hardware-specific. \textbf{False}
\end{enumerate}
 
\exercise{}{0}
\solution
\begin{itemize}
    \item[a)]The triangle is scaled with factors 3 and -1 in the x- and y-coordinate, respectively, rotated (clockwise) by 90$ \degree $ around the z-coordinate, and translated by distance bx = 2 units in the direction of the x-coordinate. The transformations are executed in the given order.
    \begin{itemize}
        \item Derive the transformation matrix of each transformation step in homogeneous coordinates.
        \begin{center}
            $ S_z(3)=
            \begin{bmatrix}
            3 & 0 & 0 & 0 \\
            0 & 1 & 0 & 0 \\
            0 & 0 & 1 & 0 \\
            0 & 0 & 0 & 1
            \end{bmatrix} $
            \end{center}
            \begin{center}
            $ S_y(-1)=
            \begin{bmatrix}
            1 & 0 & 0 & 0 \\
            0 & -1 & 0 & 0 \\
            0 & 0 & 1 & 0 \\
            0 & 0 & 0 & 1
            \end{bmatrix}$
            \end{center}
            \begin{center}
            $ R_y(\theta)=
            \begin{bmatrix}
            cos\theta & -sin\theta & 0 & 0 \\
            sin\theta & cos\theta & 0 & 0 \\
            0 & 0 & 1 & 0 \\
            0 & 0 & 0 & 1
            \end{bmatrix}$
            \\ $ \implies \theta = -\pi / 2 $
            \end{center}
            \begin{center}
            $ R_y(-\pi / 2)=
            \begin{bmatrix}
            0 & 1 & 0 & 0 \\
            -1 & 0 & 0 & 0 \\
            0 & 0 & 1 & 0 \\
            0 & 0 & 0 & 1
            \end{bmatrix}$
            \end{center}
            $ b_x = 2 $
            \begin{center}
            $ T_x(b_x)=
            \begin{bmatrix}
            1 & 0 & 0 & b_x \\
            0 & 1 & 0 & 0 \\
            0 & 0 & 1 & 0 \\
            0 & 0 & 0 & 1
            \end{bmatrix}$
            \end{center}
            \begin{center}
            $ T_x(2)=
            \begin{bmatrix}
            1 & 0 & 0 & 2 \\
            0 & 1 & 0 & 0 \\
            0 & 0 & 1 & 0 \\
            0 & 0 & 0 & 1
            \end{bmatrix}$
        \end{center}
        \newpage
        \item Compute the combined transformation matrix in homogeneous coordinates.
        \begin{center}
            $ T_x(2)R_y(-\pi/2)S_y(-1)S_x(3) = $
        \end{center}
        \begin{center}
            $ \begin{bmatrix}
            1 & 0 & 0 & 2 \\
            0 & 1 & 0 & 0 \\
            0 & 0 & 1 & 0 \\
            0 & 0 & 0 & 1
            \end{bmatrix}
            \begin{bmatrix}
            0 & 1 & 0 & 0 \\
            -1 & 0 & 0 & 0 \\
            0 & 0 & 1 & 0 \\
            0 & 0 & 0 & 1
            \end{bmatrix}
            \begin{bmatrix}
            1 & 0 & 0 & 0 \\
            0 & -1 & 0 & 0 \\
            0 & 0 & 1 & 0 \\
            0 & 0 & 0 & 1
            \end{bmatrix}
            \begin{bmatrix}
            3 & 0 & 0 & 0 \\
            0 & 1 & 0 & 0 \\
            0 & 0 & 1 & 0 \\
            0 & 0 & 0 & 1
            \end{bmatrix} = $
        \end{center}
        \begin{center}
            $ = \begin{bmatrix}
            1 & 0 & 0 & 2 \\
            0 & 1 & 0 & 0 \\
            0 & 0 & 1 & 0 \\
            0 & 0 & 0 & 1
            \end{bmatrix}
            \begin{bmatrix}
            0 & 1 & 0 & 0 \\
            -1 & 0 & 0 & 0 \\
            0 & 0 & 1 & 0 \\
            0 & 0 & 0 & 1
            \end{bmatrix}
            \begin{bmatrix}
            3 & 0 & 0 & 0 \\
            0 & -1 & 0 & 0 \\
            0 & 0 & 1 & 0 \\
            0 & 0 & 0 & 1
            \end{bmatrix} = $
        \end{center}
        \begin{center}
            $ = \begin{bmatrix}
            1 & 0 & 0 & 2 \\
            0 & 1 & 0 & 0 \\
            0 & 0 & 1 & 0 \\
            0 & 0 & 0 & 1
            \end{bmatrix}
            \begin{bmatrix}
            0 & -1 & 0 & 0 \\
            -3 & 0 & 0 & 0 \\
            0 & 0 & 1 & 0 \\
            0 & 0 & 0 & 1
            \end{bmatrix} = $
        \end{center}
        \begin{center}
            $ = \begin{bmatrix}
            0 & -1 & 0 & 2 \\
            -3 & 0 & 0 & 0 \\
            0 & 0 & 1 & 0 \\
            0 & 0 & 0 & 1
            \end{bmatrix} $
        \end{center}
        \item Apply the combined transformation matrix to triangle.
        \begin{center}
            $ = \begin{bmatrix}
            0 & -1 & 0 & 2 \\
            -3 & 0 & 0 & 0 \\
            0 & 0 & 1 & 0 \\
            0 & 0 & 0 & 1
            \end{bmatrix}
            \begin{bmatrix}
            3 & 2 & 1 \\
            0 & 0 & 1 \\
            2 & 2 & 2 \\
            1 & 1 & 1
            \end{bmatrix} = $
        \end{center}
        \begin{center}
            $ = \begin{bmatrix}
            2 & 2 & 1 \\
            -9 & -6 & -3 \\
            2 & 2 & 2 \\
            1 & 1 & 1
            \end{bmatrix} $
        \end{center}
    \end{itemize}
    \item[b)] The triangle is projected into the given image plane using perspective projections.
    \begin{itemize}
        \item Compute the projection matrix A in homogeneous coordinates for the given example. Leave the depth-related components a = b = 1.
        \begin{center}
            $ A = \begin{bmatrix}
            h & 0 & 0 & 0 \\
            0 & h & 0 & 0 \\
            0 & 0 & a & b \\
            0 & 0 & 1 & 0
            \end{bmatrix} $
        \end{center}
        h = 1; a = 1; b = 1
        \begin{center}
            $ A = \begin{bmatrix}
            1 & 0 & 0 & 0 \\
            0 & 1 & 0 & 0 \\
            0 & 0 & 1 & 1 \\
            0 & 0 & 1 & 0
            \end{bmatrix} $
        \end{center}
        \item Apply the projection matrix A to the triangle and give the results to 2-D Cartesian screen coordinates, with the origin of the screen at (0, 0, 1). Remember that a projection takes two stages!
        \begin{center}
            $\begin{bmatrix}
            1 & 0 & 0 & 0 \\
            0 & 1 & 0 & 0 \\
            0 & 0 & 1 & 1 \\
            0 & 0 & 1 & 0
            \end{bmatrix} 
            \begin{bmatrix}
            2 & 2 & 1 \\
            -9 & -6 & -3 \\
            2 & 2 & 2 \\
            1 & 1 & 1
            \end{bmatrix} = $
        \end{center}
        \begin{center}
            $ = \begin{bmatrix}
            2 & 2 & 1 \\
            -9 & -6 & -3 \\
            3 & 3 & 3 \\
            2 & 2 & 2
            \end{bmatrix} $
        \end{center}
        \begin{center}
            $ \frac{1}{2} \begin{bmatrix}
            2 & 2 & 1 \\
            -9 & -6 & -3 \\
            3 & 3 & 3 \\
            2 & 2 & 2
            \end{bmatrix} = $
        \end{center}
        \begin{center}
            $ = \begin{bmatrix}
            1 & 1 & 0.5 \\
            -4.5 & -3 & -1.5 \\
            1.5 & 1.5 & 1.5 \\
            1 & 1 & 1
            \end{bmatrix} $
        \end{center}
        \begin{center}
            $ p_1 = (1, -4.5) $ \\ 
            $ p_2 = (1, -3) $ \\ 
            $ p_3 = (0.5, -1,5) $
        \end{center}
        According to these coordinates we won't be able to see the triangle on the defined screen.
    \end{itemize}
\end{itemize}

\exercise{}{0}
\solution
\begin{itemize}
    \item Compute matrix \textbf{$ H_{ry} $} which will allow you to apply a rotation about the y-axis by angle $ \theta  = 30 \degree $ and matrix \textbf{$ T_x $} which will allow you to translate the cube along the x-axis by $ b_x = 0.5 $ units in homogeneous coordinates. \\
    The vertices of the 3D cube:
    \begin{center}
        $ \langle (-1, 1, -1), (1, 1, -1), (1, -1, -1), (-1, -1, -1), (-1, 1, 1), (1, 1, 1), (1, -1, 1), (-1, -1, 1) \rangle $
    \end{center}
    \begin{center}
        $ A = \begin{bmatrix}
        -1 & 1 & 1 & -1 & -1 & 1 & 1 & -1 \\
        1 & 1 & -1 & -1 & 1 & 1 & -1 & -1 \\
        -1 & -1 & -1 & -1 & 1 & 1 & 1 & 1 \\
        1 & 1 & 1 & 1 & 1 & 1 & 1 & 1
        \end{bmatrix} $
    \end{center}
    \begin{center}
        $ H_{ry} = \begin{bmatrix}
        cos \theta & 0 & sin \theta & 0 \\
        0 & 1 & 0 & 0 \\
        -sin \theta & 0 & cos \theta & 0 \\
        0 & 0 & 0 & 1
        \end{bmatrix} $
    \end{center}
    \begin{center}
        $ \theta = \pi / 6 $
    \end{center}
    \begin{center}
        $ H_{ry} = \begin{bmatrix}
        \sqrt{3}/2 & 0 & 1/2 & 0 \\
        0 & 1 & 0 & 0 \\
        -1/2 & 0 & \sqrt{3}/2 & 0 \\
        0 & 0 & 0 & 1
        \end{bmatrix} $
    \end{center}
    Computing $ T_x $ translating along the x axis by $ b_x = 0.5 $
    \begin{center}
        $ T_x = \begin{bmatrix}
        1 & 0 & 0 & b_x \\
        0 & 1 & 0 & 0 \\
        0 & 0 & 1 & 0 \\
        0 & 0 & 0 & 1
        \end{bmatrix} $
    \end{center}
    \begin{center}
        $ T_x = \begin{bmatrix}
        1 & 0 & 0 & 0.5 \\
        0 & 1 & 0 & 0 \\
        0 & 0 & 1 & 0 \\
        0 & 0 & 0 & 1
        \end{bmatrix} $
    \end{center}
    \item Apply the combined transformation matrix to the cube and compute the new vertex coordinates.
    \begin{center}
        $ [T_x][h_ry] = \begin{bmatrix}
        1 & 0 & 0 & 0.5 \\
        0 & 1 & 0 & 0 \\
        0 & 0 & 1 & 0 \\
        0 & 0 & 0 & 1
        \end{bmatrix}
        \begin{bmatrix}
        \sqrt{3}/2 & 0 & 1/2 & 0 \\
        0 & 1 & 0 & 0 \\
        -1/2 & 0 & \sqrt{3}/2 & 0 \\
        0 & 0 & 0 & 1
        \end{bmatrix} = $
    \end{center}
    \begin{center}
        $ = \begin{bmatrix}
        \sqrt{3}/2 & 0 & 1/2 & 1/2 \\
        0 & 1 & 0 & 0 \\
        -1/2 & 0 & \sqrt{3}/2 & 0 \\
        0 & 0 & 0 & 1
        \end{bmatrix}$
    \end{center}
    \begin{center}
        $ [T_x][H_{ry}][A] = \begin{bmatrix}
        \sqrt{3}/2 & 0 & 1/2 & 1/2 \\
        0 & 1 & 0 & 0 \\
        -1/2 & 0 & \sqrt{3}/2 & 0 \\
        0 & 0 & 0 & 1
        \end{bmatrix}
        \begin{bmatrix}
        -1 & 1 & 1 & -1 & -1 & 1 & 1 & -1 \\
        1 & 1 & -1 & -1 & 1 & 1 & -1 & -1 \\
        -1 & -1 & -1 & -1 & 1 & 1 & 1 & 1 \\
        1 & 1 & 1 & 1 & 1 & 1 & 1 & 1
        \end{bmatrix} = $
    \end{center}
    \begin{center}
        $ = \begin{bmatrix}
        -0.866 & 0.866 & -0.866 & -1.366 & 0.134 & 1.866 & 1.866 & -0.134 & 0.5 \\
        1 & 1 & -1 & -1 & 1 & 1 & -1 & -1 & 0 \\
        -0.366 & -1.366 & -1.366 & -0.366 & 1.366 & 0.366 & 0.366 & 1.366 & 0 \\
        1 & 1 & 1 & 1 & 1 & 1 & 1 & 1 & 1 
        \end{bmatrix}$
    \end{center}
    \item Assuming $ r = -l $ and $ t = -b $ pick appropriate camera frustrum parameters $ r, l, t, b, f $ and  $ n $ to complete the matrix and ensure that all points of the cube are in the frustrum ($ \equiv $ they will remain in the unit cube after the projection). Remark: frustrum in case of orthographic projection is much simpler than in case of perspective projection!
    \begin{center}
        $ P_{\perp} = \begin{bmatrix}
        \frac{2}{r-1} & 0 & 0 & -\frac{r+1}{r-1} \\
        0 & \frac{2}{t-b} & 0 & -\frac{t+b}{t-b} \\
        0 & 0 & -\frac{2}{f-n} & -\frac{f+n}{f-n} \\
        0 & 0 & 0 & 1
        \end{bmatrix} = 
        \begin{bmatrix}
        \frac{1}{r} & 0 & 0 & 0 \\
        0 & \frac{1}{t} & 0 & 0 \\
        0 & 0 & -\frac{2}{f-n} & -\frac{f+n}{f-n} \\
        0 & 0 & 0 & 1
        \end{bmatrix}$
    \end{center}
    \item Assuming the same camera position as in the previous exercise, carry out the projection of the transformed cube's vertices.
    \item If the cube was to be translated along the z-axis by 1 unit towards the camera, would you have to update your frustrum parameters $ r, l, t $ and $ b $? Explain your answer.
\end{itemize}

\end{document}